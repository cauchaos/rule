\documentclass{oblivoir}

% 참고: 서식을 만들 때 법제체의 법령입안심사기준을 참고함.

% 피드백용 명령에 이용됨.
\usepackage[dvipsnames]{xcolor}
\usepackage[normalem]{ulem}
% 장, 조 스타일을 만들 때 이용 됨
\usepackage{titlesec}
% 항, 호, 목을 만들 때 이용 됨.
\usepackage{enumitem}
% 바닥글, 머릿글 설정에 이용 됨.
\usepackage{fancyhdr}
\usepackage{lastpage}

% 제목
\title{ChAOS 회칙}
\date{2018년 11월 제정}

% 머릿글
\fancyhead[L]{ChAOS 회칙}
\fancyhead[C]{}
\fancyhead[R]{}
\renewcommand{\headrulewidth}{1pt}

% 바닥글
\fancyfoot[L]{\url{https://cauchaos.github.io}}
\fancyfoot[C]{}
\fancyfoot[R]{\thepage / \pageref{LastPage} 페이지}
\renewcommand{\footrulewidth}{0.5pt}

% 머릿글과 바닥글 설정 적용
\fancypagestyle{plain}{
    \pagestyle{fancy}
} % 첫 페이지에 적용
\pagestyle{fancy} % 2+ 페이지에 적용

% 회칙 제목 스타일 변경
\pretitle{\begin{center}\Huge}
\posttitle{\end{center}}
\predate{\begin{center}\Large}
\postdate{\end{center}}

% 저자(\author) 관련 공백 삭제
\preauthor{}
\postauthor{}
\renewcommand{\maketitlehookb}{}

% 장
\titleformat{\chapter}[hang]
{\normalfont\LARGE\bfseries}{제 \thechapter{}\chaptertitlename.}{0.5em}{}

% 장 위아래의 공백 조정
\titlespacing{\chapter}{0pt}{1em}{0.25em}

% 조
\newcommand{\wrapwithparenthesis}[1]{(#1)}
\titleformat{\section}
{\normalfont\large\bfseries}{제 \thesection{}조 }{0.5em}{\wrapwithparenthesis}

% 가나다 넘버링
\makeatletter
\def\ganadatext#1{\expandafter\@ganadatext\csname c@#1\endcsname}
\def\@ganadatext#1{\ifcase#1\or 가\or 나\or 다\or 라\or 마\or 바\or 사\or 아\or 자\or 차\or 카\or 타\or 파\or 하\fi}
\makeatother
\AddEnumerateCounter{\ganadatext}{\@ganadatext}{하}

% 항 
\setlist[enumerate, 1]{label=\large\protect\textcircled{\small\arabic*}, ref={\arabic*}}%ref={제\thesection조제\arabic*항}}

% 호
\setlist[enumerate, 2]{label={\arabic*.}, ref={\arabic*}}% ref={제\thesection조제\arabic{enumi}항\arabic*목}}

% 목
\setlist[enumerate, 3]{label={\ganadatext*.}, ref={\arabic*}}% ref={제\thesection조제\arabic{enumi}항\arabic{enumii}목\arabic*목}}
\setlist[enumerate, 4]{label={\arabic*)}, ref={\arabic*}}% ref={제\thesection조제\arabic{enumi}항\arabic{enumii}목\arabic{enumiii}목\arabic*목}}
\setlist[enumerate, 5]{label={\ganadatext*)}, ref={\arabic*}}% ref={제\thesection조제\arabic{enumi}항\arabic{enumii}목\arabic{enumiii}목\arabic{enumiii}목\arabic*목}}

% % 회칙 피드백시 사용되는 명령
% \newcommand{\feedback}[1]{\textcolor{Dandelion}{#1}}
% \newcommand{\addfeedback}[1]{\textcolor{Green}{#1}}
% \newcommand{\delfeedback}[1]{\textcolor{Maroon}{\sout{#1}}}

% 회칙 내용 시작
\begin{document}
\maketitle

\chapter{총칙}

\section{명칭}
\begin{enumerate}
    \item 본 학회의 한글 명칭은 `중앙대학교 알고리즘 학회'이다.
    \item 본 학회의 영문 명칭은 `Chung-ang university Algorithm Organization and Society'이다.
    \item 본 학회의 약칭은 `ChAOS'이며, 한글의 경우에는 `카오스'라 표기한다.
\end{enumerate}

\section{목표}
\begin{enumerate}
    \item 첫 번째 목표 \\ 본 학회가 생각하는 더 나은 개발자는 다음과 같은 조건을 가진다.
          \begin{enumerate}
              \item 첫 번째로 언어를 자유자재로 다룰 수 있는 숙련도를 지니고 있어야 한다.
              \item 두 번째로 자료구조와 알고리즘에 대한 깊은 이해도를 지니고 있어야 한다.
          \end{enumerate}
          이러한 생각 아래 본 학회는 회원의 프로그래밍 언어와 자료구조, 알고리즘에 대한 이해와 적용을 증진시킨다.
    \item 두 번째 목표 \\ 본 학회는 학부와 학회 내 학우의 알고리즘에 대한 관심과 흥미를 증진시킨다.
\end{enumerate}

\chapter{회원}
\label{회원 장}

\section{구성}
\begin{enumerate}
    \item 본 학회의 회원은 정회원과 준회원으로 구성된다.
\end{enumerate}

\section{자격}
\label{자격 조항}
\begin{enumerate}
    \item \label{정회원 자격 1}이어지는 두 분기 내에 세 개 이상 회장이 인정하는 대회에 참여할 시 정회원의 자격을
          가질 수 있다.
    \item \label{정회원 자격 2}한 분기 내에 교육 부문 중 한 가지 이상에 참여하여 2/3 이상 이수할 시 정회원의 자격을 가질 수 있다.
    \item \label{정회원 자격 3}매 분기 시작 시 회장이 공표하는 기준을 충족한 자는 정회원의 자격을 가질 수 있다.
    \item 중앙대학교에 재학하는 자라면 누구든지 회장에게 가입 의사를 표시하여 준회원의 자격을 가질 수 있다.
\end{enumerate}

\section{권리}
\begin{enumerate}
    \item 정회원
          \begin{enumerate}
              \item 회장에 대한 선거권과 피선거권을 가진다.
              \item 회칙 개정을 비롯한 의사 결정 과정에 참여할 수 있는 권리를 지닌다.
              \item 본 학회가 실시하는 모든 활동에 참여할 수 있는 권리를 지닌다.
          \end{enumerate}
    \item 준회원
          \begin{enumerate}
            \item 본 학회가 실시하는 활동 중 회장이 인정하는 일부 활동에 참여할 수 있는 권리를 지닌다.
          \end{enumerate}
\end{enumerate}

\section{의무}
\begin{enumerate}
    \item 본 학회의 회원은 본 학회의 활동에 성실하게 참여할 의무를 가진다.
    \item 본 학회의 정회원은 본 학회의 운영에 필요한 회비를 납부해야 한다.
\end{enumerate}

\section{휴면}
\begin{enumerate}
    \item 정회원은 병역, 휴학, 회장이 인정하는 불가피한 사유로 일부 기간 동안 활동이 불가능할
          경우 휴면할 수 있다.
    \item 휴면 신청 시 회장에게 휴면하는 기간과 사유를 알려야 한다.
    \item 휴면 중 휴면 기간 또는 사유에 변동이 발생할 경우 변경 사항을 알려야 한다.
    \item 휴면 기간 동안 기존 등급이 변동되지 않고 유지된다.
\end{enumerate}

\section{은퇴}
\begin{enumerate}
    \item
          정회원은 졸업, 회장이 인정하는 불가피한 사유로 활동이 불가능한 경우 은퇴할 수 있다.
    \item
          은퇴 이후 기존의 권한과 의무는 소멸하나, 회장이 인정하는 본 학회의 활동에 참여할
          수 있다.
\end{enumerate}

\section{탈퇴}
\begin{enumerate}
    \item  탈퇴를 원하는 회원은 회장에게 탈퇴 의사를 표시한 후, 자유롭게 탈퇴할 수 있다.
    \item  탈퇴 이후 기존의 권한과 의무는 모두 소멸한다.
    \item  탈퇴 이후 준회원으로 재가입할 수 있다.
\end{enumerate}

\section{제명}
\begin{enumerate}
    \item  정회원은 다른 회원의 제명 안건을 정모에 상정할 수 있으며, 정모의 출석한 정회원의 과반의 찬성으로 가결된다.
\end{enumerate}

\section{강등}
\begin{enumerate}
    \item 정회원의 자격을 가진 후 이어지는 네 분기 동안 `제 \ref{회원 장}장 제 \ref{자격 조항}조 \ref{정회원 자격 1}항, \ref{정회원 자격 2}항, 혹은 \ref{정회원 자격 3}항'의 자격 요건을 갖추지 못할 시 회장의 판단 하에 준회원으로 강등된다.
\end{enumerate}

\chapter{회장}

\section{회장}
\begin{enumerate}
    \item  학회의 대표로서 본 학회 운영의 최종 결정 권한을 가지는 정회원이다.
    \item  본 학회의 회원과 재정을 관리하며 본 학회의 일체의 활동을 총괄한다.
    \item  부 회장을 비롯한 정회원으로 이루어진 임원단을 구성하여 권한을 분할하거나 위임할
          수 있다.
\end{enumerate}

\section{후보}
\begin{enumerate}
    \item  아래 각 호를 모두 충족하는 자는 학회의 회장 선거 후보로 출마할 수 있다.
    \begin{enumerate}
        \item 본 학회 내 정회원으로 재적 중인 자
        \item 중앙대학교 소프트웨어학부에 재학 중인 자
        \item 휴면 중이 아니며 이어지는 네 분기 동안 휴면하지 않을 자
    \end{enumerate}
\end{enumerate}

\section{선거}
\begin{enumerate}
    \item  보통 · 평등 · 직접 · 비밀 선거에 의하여 선출한다.
    \item  선거일로 7일 이전 공고된 정모에서 출석한 정회원의 과반 득표로 당선된다.
    \item  부득이한 경우 온라인 투표를 7일 이상 진행하며 정회원의 재적 인원 과반 참여와 참여
          인원 과반 득표로 당선된다.
    \item  회장의 정기 선출 시기는 4분기 말로 한다.
\end{enumerate}

\section{임기}
\begin{enumerate}
    \item  임기는 1분기 시작일로부터 4분기 종료일까지로 한다.
    \item  \label{president-must-appoint-the acting}임기 중 불가피한 사유로 직무를 수행하지 못하게 될 경우 회장이 회장 권한 대행을 지
          명하거나, 후임 회장을 선출해야 한다.
\end{enumerate}

\chapter{운영}

\section{정모}
\begin{enumerate}
    \item 정모는 본 학회의 회원들의 정기 모임으로 각종 안건을 처리한다.
    \item 다음 정모의 일시는 정모 1주일 전에 공지한다.
    \item 정모의 장소는 회장의 결정에 의하여 정모 3일 전까지 공지한다.
    \item 한 학기에 두 번 이상 필수적으로 진행한다.
\end{enumerate}

\section{분기}
\begin{enumerate}
    \item  본 학회는 당해의 겨울방학을 1분기, 1학기를 2분기, 여름방학을 3분기, 2학기를 4분기
          로 지정한다.
\end{enumerate}

\section{회계}
\begin{enumerate}
    \item  회장은 본 학회의 운영에 필요하다고 판단할 때마다 회비를 회원에게 걷을 수 있다.
    \item  회장은 정모에서 정한 사용 목적과 제한 내에서 재정을 집행해야 한다.
    \item  회장은 회비를 걷을 경우 회계 내역을 2분기 말과 4분기 말 정모에서 회원에게 공고해야 한다.
\end{enumerate}

\section{구조}
\begin{enumerate}
    \item  본 학회의 사업을 비롯한 구조는 당해의 회장이 정한다.
\end{enumerate}

\section{기타}
\begin{enumerate}
    \item  그 외의 운영에 관한 모든 사항은 정모를 통해 결정한다.
\end{enumerate}

\section{개정}
\begin{enumerate}
    \item  본 회칙의 개정은 정모에서 출석한 정회원의 2/3 이상 찬성과 회장의 승인으로 가결한
          다.
    \item  개정 당일 개정된 회칙을 모든 정회원에게 공고하고, 7일 동안의 이의 제기 기간 동안
          이의 제기가 없을 시 효력이 발생한다.
\end{enumerate}
\end{document}